\section{Opis wymagań funkcjonalnych}
Ten rozdział opisuje wymagania funkcjonalne projektowanego komunikatora. Głównych
aktorów występujących w systemie przedstawia Tabela \ref{tab:aktorzy}. Aktorzy
zostali przedstawieni w sposób hierarchiczny, względem rosnących uprawnień.
Dodatkowo, dla użytkowników zalogowanych na serwerze, uprawnienia są dziedziczone.

\begin{table}[!hbp]
	\centering
	\caption{Aktorzy występujący w systemie.}
	\label{tab:aktorzy}
	\begin{tabular}{|r|l|} \hline
		\multicolumn{1}{|c|}{\textbf{Lp.}} & \multicolumn{1}{|c|}{\textbf{Aktor}} \\ \hline
		1. & Użytkownik niezalogowany \\ \hline \hline
		2. & Użytkownik zalogowany \\ \hline
		3. & Operator kanału \\ \hline
		4. & Administrator serwera \\ \hline
	\end{tabular}
\end{table}

Wymagania funkcjonalne projektowanego komunikatora zaprezentowano w Tabeli
\ref{tab:wymfun}. Zebrane wymagania dotyczą poszczególnych obszarów działania
komunikatora i powiązane są z podanymi powyżej aktorami (użytkownikami systemu).

\begin{longtabu} to \linewidth {|r|p{3.45cm}|X|p{2.6cm}|}
	\caption{Wymagania funkcjonalne aplikacji.}
	\label{tab:wymfun} \\
	
	\hline
	\rowfont\bfseries Lp. & Obszar działania & Wymagania funkcjonalne & Aktor \\ \hline
	\endhead
	\endfoot
	\endlastfoot

	1.&
	Autoryzacja &
	Każdy użytkownik musi zarejestrować się na serwerze. Wymagane do
	rejestracji są: nazwa użytkownika, adres e-mail i hasło. &
	Użytkownik niezalogowany \\ \cline{1-1} \cline{3-4}
	
	2.&
	&
	Użytkownik loguje się na serwerze korzystając z własnego loginu i hasła &
	Użytkownik niezalogowany\\ \hline
	
	3.&
	Operacje na kanałach &
	Na serwerze może być prowadzone wiele rozmów jednocześnie. Użytkownik
	dołącza do kanału, na którym chce prowadzić rozmowę. &
	Użytkownik standardowy, Baza Danych, Serwer\\ \hline
	
	4.&
	Nadawanie dźwięku &
	Nadawanie głosu rozpoczyna się z chwilą przyciśnięcia zdefiniowanego
	przycisku lub w trybie ciągłym, z chwilą połączenia do kanału. Transmisja
	odbywa się do serwera usługi. &
	Użytkownik standardowy, Serwer\\ \hline
	
	5.&
	Odbieranie dźwięku & Serwer wysyła do klientów dźwięk. &
	Użytkownik standardowy, Serwer\\ \hline
	
	6.&
	Usuwanie użytkownika z kanału & Użytkowników naruszających zasady ustalone
	przez administratora serwera i/lub kanału można usunąć z konwersacji. &
	Moderator / Administrator\\ \hline
	
	7.&
	Banowanie użytkownika &
	Użytkowników rażąco naruszających zasady można całkowicie zablokować. &
	Administrator\\ \hline
	
	8.&
	Utworzenie kanału tymczasowego &
	Użytkownik może stworzyć własny kanał. Przy tworzeniu może zdefiniować
	przepływność kodeka używana na kanale. Kanał jest usuwany automatycznie w
	momencie opuszczenia go przez ostatniego użytkownika. &
	Użytkownik standardowy\\ \hline
	
	9.&
	Utworzenie kanału permanentnego &
	Administrator może przekształcić kanał tymczasowy na permanentny na prośbę
	użytkownika. &
	Administrator, Użytkownik standardowy \\ \hline
\end{longtabu}
