\section {Protokół komunikacyjny}
W tym rozdziale przedstawiono protokół komunikacyjny, który będzie stosowany
do komunikacji pomiędzy klientem, a serwerem. Dla łatwiejszej analizy
przesyłanych pakietów wykorzystany zostanie protokół tekstowy wysyłany w formie
pakietów TCP, częściowo bazujący na protokole SIP.

Komunikat składa się z linii wprowadzającej zawierającej informację o~wysyłanym
komunikacie oraz wersji protokołu. Następnie znajdują się pola nadawcy
i~odbiorcy. Na końcu mogą znaleźć się opcjonalne pola, które zależne będą
od rodzaju komunikatu. Komunikat zawsze kończy się dwoma znakami nowej
linii w systemie Windows (kody ASCII \texttt{13} i \texttt{10})

Dla przejrzystości, komunikaty żądań oraz odpowiedzi opisano w~osobnych
poddziałach przedstawionych poniżej.

\subsection{Komunikaty żądań}
W~tej sekcji przedstawiono żądania wchodzące w skład projektowanego protokołu.
Spis komunikatów, wraz z krótkim odpisem zaprezentowano w Tabeli \ref{tab:requests}

\begin{longtabu} to \linewidth {|>{\raggedright}r|>{\raggedright}p{2.65cm}|>{\raggedright\arraybackslash}X|}
	\caption{Wymagania funkcjonalne aplikacji.}
	\label{tab:requests} \\
	
	\hline
	\rowfont\bfseries Lp. & Komunikat & Opis \\ \hline
	\endhead
	\endfoot
	\endlastfoot
	
	1. & REGISTER & Rejestracja na serwerze za pomocą podanego loginu i~hasła   \\ \hline
	2. & CONNECT & Logowanie na serwerze przy użyciu loginu i~hasła  \\ \hline
	3. & DISCONNECT & Rejestracja na serwerze za pomocą podanego loginu i~hasła   \\ \hline
	4. & OPTIONS & Wyświetla informacje o~wybranym kanale. \\ \hline
	5. & JOIN & Próba dołączenia do kanału. \\ \hline
	6. & LEAVE & Opuszczenie rozmowy. \\ \hline
	7. & CHKICK & Wyrzucenie użytkownika z kanału. \\ \hline
	8. & SRKICK & Wyrzucenie użytkownika z serwera. \\ \hline
\end{longtabu}

\subsection{Komunikaty odpowiedzi}
Żądania w protokole oczekują odpowiedzi. Skorzystano ze sprawdzonego
w~protokołach tekstowych schematu numerowania. Charakterystykę ogólną
przedstawiono na poniższym wykazie.
\begin{itemize}
	\item 200 -- odpowiedzi prawidłowe
	\item 400 -- błędy spowodowane przez użytkownika
	\item 500 -- błędy spowodowane przez serwer
\end{itemize}
