\section{Opis wymagań niefunkcjonalnych}
Ten rozdział opisuje wymagania niefunkcjonalne systemu. Opisują one ograniczenia oraz kryteria jakościowe, jakie spełnić musi projektowana aplikacja. Wymagania niefunkcjonalne dla projektowanych aplikacji przedstawia tabela \ref{tab:wymniefun}.

\begin{table}[!hbp]
	\centering
	\caption{Wymagania niefunkcjonalne projektu ,,Babbler''}
	\label{tab:wymniefun}
	\begin{tabular}{|r|p{\linewidth}|} \hline
		\multicolumn{1}{|c|}{\textbf{Lp.}} & \multicolumn{1}{|c|}{\textbf{Wymaganie niefunkcjonalne}} \\ \hline
		1. & Aplikacja sieciowa w architekturze klient-serwer. \\ \hline
		2. & Dla klienta minimalna wersja .NET Framework 4.5.2. \\ \hline
		3. & Dla serwera interpreter języka Python 3.5 lub kompilator C++ zgody ze standardem C++14 \\ \hline
		4. & Wykorzystanie nowoczesnego kodeka audio \textit{Opus} \\ \hline
		5. & Język programowania: głównie C\# (klient), Python lub C++ (serwer) \\ \hline
		6. & Aplikacja dostępna jedynie w języku polskim \\ \hline
		7. & System powinien zabezpieczać odpowiednie funkcjonalności przed osobami nieposiadającymi uprawnień. \\ \hline
		8. & System powinien przechowywać hasła użytkowników w sposób uniemożliwiający łatwe ich poznanie (bezpieczna funkcja skrótu) \\ \hline
		9. & Stały dostęp do Internetu z przepustowością minimum 64 kbps w obie strony \\ \hline
	\end{tabular}
\end{table}
